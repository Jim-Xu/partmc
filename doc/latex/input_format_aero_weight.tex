Read an aero\+\_\+weight from a spec file.


\begin{DoxyParams}[1]{Parameters}
\mbox{\tt in,out}  & {\em file} & Spec file.\\
\hline
\mbox{\tt in,out}  & {\em aero\+\_\+weight} & Aerosol weight.\\
\hline
\end{DoxyParams}
For efficiency the aerosol population can be weighted so that the true number distribution $n(D)$ is given by \[ n(D) = w(D) c(D) \] where $w(D)$ is a fixed weighting function, $c(D)$ is the computational (simulated) number distribution, and $D$ is the diameter. Thus a large value of $w(D)$ means that relatively few computational particles are used at diameter $D$, while a small value of $w(D)$ means that relatively many computational particles will be used at that diameter.

The aerosol weighting function is specified by the parameters\+:
\begin{DoxyItemize}
\item {\bfseries weight} (string)\+: the type of weighting function --- must be one of\+: \char`\"{}none\char`\"{} for no weighting ( $w(D) = 1$); \char`\"{}power\char`\"{} for a power-\/law weighting ( $w(D) = D^\alpha$), or \char`\"{}mfa\char`\"{} for the mass flow algorithm weighting ( $w(D) = D^{-3}$ with dependent coagulation particle removal)
\item if the {\ttfamily weight} is {\ttfamily power} then the next parameter is\+:
\begin{DoxyItemize}
\item {\bfseries exponent} (real, dimensionless)\+: the exponent $\alpha$ in the power law relationship --- setting the {\ttfamily exponent} to 0 is equivalent to no weighting, while setting the exponent negative uses more computational particles at larger diameters and setting the exponent positive uses more computational particles at smaller diameters; in practice exponents between 0 and -\/3 are most useful
\end{DoxyItemize}
\end{DoxyItemize}

See also\+:
\begin{DoxyItemize}
\item \mbox{\hyperlink{spec_file_format}{Input File Format\+: Spec File Format}} --- the input file text format 
\end{DoxyItemize}