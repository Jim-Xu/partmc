\documentclass{article}
\usepackage{amsmath}
\usepackage[margin=2cm]{geometry}
\usepackage{longtable}
\providecommand{\e}[1]{\ensuremath{\cdot 10^{#1}}}

\begin{document}

\newcommand{\rr}{\raggedright}
\newcommand{\tn}{\tabularnewline\hline}
\renewcommand{\arraystretch}{1.5}
\begin{longtable}{|l|p{5.5cm}|l|l|p{4.5cm}|}
\hline \textbf{Symbol} & \textbf{Meaning}                                                       & \textbf{Type} & \textbf{Units}                        &  \textbf{Reference}                                \tn
\hline \endhead
$a$					   & \rr exponent in size dependence of diffusive boundary layer thickness	& input			& $1$									& \rr eq.~\ref{eq:BL_thick}, value = 0.274			 \tn
$A$					   & \rr constant in slip correction formula								& input			& $1$									& \rr eq.~\ref{eq:slip_correct}, value = 1.142		 \tn
$A_{\rm D}$			   & \rr diffusional deposition area										& input			& $\rm m^2$								& \rr eq.~\ref{eq:diffusion}, value = 103			 \tn
$A_{\rm S}$			   & \rr sedimentational deposition area									& input			& $\rm m^2$								& \rr eq.~\ref{eq:sedimentation}, value = 12.6		 \tn
$b$					   & \rr constant in slip correction formula								& input			& $1$									& \rr eq.~\ref{eq:slip_correct}, value = 0.999		 \tn
$C$					   & \rr slip correction factor from continuum to free molecular regime		& function		& $1$									& \rr eq.~\ref{eq:slip_correct}						 \tn
$d_{\rm f}$			   & \rr volume fractal dimension											& input			& $1$									& \rr Table 2 from Naumann 2003						 \tn
$d_{\rm s}$			   & \rr surface fractal dimension											& input			& $1$									& \rr $d_{\rm f} \leq 2, d_{\rm s} = 3$; $2 \leq d_{\rm f} \leq 3, d_{\rm s} = \frac{6}{d_{\rm f}}$				 \tn
$D$					   & \rr translational diffusion coefficient								& function		& $\rm m^2 \, s^{-1}$					& \rr eq.~\ref{eq:Diff_coef}						 \tn
$D_{\rm 0}$			   & \rr unit translational diffusion coefficient							& input			& $\rm m^2 \, s^{-1}$					& \rr eq.~\ref{eq:Diff_coef}, value = 1 from Bunz \& Dlugi, 1991		 \tn
$D(R_{\rm me,\it i})$  & \rr translational diffusion coefficient for particle i					& function		& $\rm m^2 \, s^{-1}$					& \rr eq.~\ref{eq:Diff_coef}						 \tn
$f$					   & \rr volume filling factor												& input			& $1$									& \rr Table 2 from Naumann 2003						 \tn
$g$					   & \rr gravitational acceleration											& constant		& $\rm m \, s^{-2}$						& \rr \verb+const%g+			 					 \tn
$h_{\rm KR}$		   & \rr Kirkwood-Riseman ratio												& function		& $1$									& \rr eq.~\ref{eq:h_KR}			 					 \tn
$k$					   & \rr Boltzmann constant													& constant		& $\rm m^2 \, kg \, s^{-2} \, K^{-1}$	& \rr \verb+const%boltzmann+	 					 \tn
$k_{\rm D}$			   & \rr prefactor in size dependence of diffusive boundary layer thickness	& input			& $\rm m$								& \rr eq.~\ref{eq:BL_thick}, value = 0.005			 \tn
$l$					   & \rr mean free path of carrier gas molecules							& input			& $\rm m$								& \rr value = 6.65\e{-8}, from excel file			 \tn
$m$					   & \rr particle mass														& input			& $\rm kg$								& \rr eq.~\ref{eq:m_Rgeo}		 					 \tn
$N$					   & \rr number of monomers in a cluster									& function		& $1$									& \rr eq.~\ref{eq:Num_monomer}	 					 \tn
$Q$					   & \rr constant in slip correction formula								& input			& $1$									& \rr eq.~\ref{eq:slip_correct}, value = 0.588		 \tn
$R_{\rm 0}$			   & \rr radius of primary particles										& input			& $\rm m$								& \rr Table 2 from Naumann 2003	 					 \tn
$R_{\rm ae}$		   & \rr aerodynamic radius													& function		& $\rm m$								& \rr eq.~\ref{eq:R_ae}			 					 \tn
$R_{\rm eff}$		   & \rr effective radius to be used with slip correction					& function		& $\rm m$								& \rr eq.~\ref{eq:R_eff}							 \tn
$R_{\rm geo}$		   & \rr geometric radius of fractal particles								& function		& $\rm m$								& \rr eq.~\ref{eq:R_geo}							 \tn
$R_{\rm geo}^{\ast}$   & \rr $R_{\rm geo}$~$\slash$~$R_{\rm 0}$									& function		& $1$									& \rr 												 \tn
$R_{\rm m}$		   	   & \rr mass equivalent radius												& function		& $\rm m$								& \rr eq.~\ref{eq:R_m}								 \tn
$R_{\rm m,\it i}$	   & \rr mass equivalent radius for particle i								& function		& $\rm m$								& \rr 												 \tn
$R_{\rm me}$		   & \rr mobility equivalent radius											& function		& $\rm m$								& \rr eq.~\ref{eq:R_me}								 \tn
$R_{\rm me,c}$		   & \rr continuum regime mobility equivalent radius						& function		& $\rm m$								& \rr eq.~\ref{eq:R_me_c}							 \tn
$R_{\rm me,\it i}$	   & \rr mobility equivalent radius for particle i							& function		& $\rm m$								& \rr 												 \tn
$S_{\rm acc}$		   & \rr accessible particle surface										& function		& $\rm m^2$								& \rr eq.~\ref{eq:S_acc}							 \tn
$T$                    & \rr absolute temperature               			                    & prescribed    & $\rm K$                               & \rr value = 296 				                     \tn
$V$					   & \rr aerosol chamber volume												& input			& $\rm m^3$								& \rr value = 84.3 									 \tn
$z$					   & \rr scaling factor in eq.~\ref{eq:S_acc} 								& input			& $1$									& \rr value = 1										 \tn
$\alpha_i^D$		   & \rr diffusional deposition coefficient for particle i					& function		& $\rm s^{-1}$							& \rr eq.~\ref{eq:diffusion}						 \tn
$\alpha_i^S$		   & \rr sedimentational deposition coefficient for particle i				& function		& $\rm s^{-1}$							& \rr eq.~\ref{eq:sedimentation}					 \tn
$\gamma$			   & \rr scaling exponent in eq.\ref{eq:S_acc} 	 							& input			& $1$									& \rr value = 0.86									 \tn
$\Gamma(..)$		   & \rr gamma function														& function		& $1$									& \rr												 \tn
$\delta_{\rm D}$	   & \rr diffusional boundary layer thickness								& function		& $\rm m$								& \rr eq.~\ref{eq:BL_thick}							 \tn
$\eta$				   & \rr gas viscosity														& constant		& $\rm kg \, m^{-1} \, s^{-1}$			& \rr \verb+const%air_dyn_visc+						 \tn
$\pi$				   & \rr Pi																	& constant		& $1$									& \rr \verb+const%pi+			 					 \tn
$\phi$				   & \rr constant in eq.\ref{eq:R_eff_simplified_2}							& constant		& 										& \rr eq.~\ref{eq:phi}								 \tn
$\xi$				   & \rr scaling factor in eq.\ref{eq:R_me_c_anal}							& function		& $1$									& \rr eq.~\ref{eq:xi}								 \tn
$\rho$				   & \rr material density of primary particles (black carbon)				& constant		& $\rm kg \, m^{-3}$					& \rr \verb+aero_data%density(BC)+					 \tn
$\rho_{\rm 0}$		   & \rr material density of reference particles (latex sphere)				& input			& $\rm kg \, m^{-3}$					& \rr eq.~\ref{eq:R_ae}, value = 940				 \tn
\end{longtable}

\newpage

\begin{align}
  \alpha_i^D &= \frac{D(R_{\rm me,\it i}) A_{\rm D}}{\delta_{\rm D} V} \label{eq:diffusion} \displaybreak[0] \\ 
  \alpha_i^S &= \frac{4 \pi \rho R_{\rm m,\it i}^3 g D(R_{\rm me,\it i}) A_{\rm S}}{3 k T V} \label{eq:sedimentation} \displaybreak[0] \\
  \delta_{\rm D} &= k_{\rm D} \left(\frac{D}{D_{\rm 0}}\right)^{\rm a} \label{eq:BL_thick} \displaybreak[0] \\
  D &= \frac{k T C(R_{\rm me})}{6 \pi \eta R_{\rm me}} = \frac{k T C(R_{\rm eff})}{6 \pi \eta R_{\rm me,c}} \label{eq:Diff_coef} \displaybreak[0] \\
  %C(R_{\rm eff}) &= 1 + A \frac{l}{R_{\rm eff}} + Q \frac{l}{R_{\rm eff}} \exp\biggl(-b \frac{R_{\rm eff}}{l}\biggr)
%  \label{eq:slip_correct} \displaybreak[0] \\
  C(R) &= 1 + A \frac{l}{R} + Q \frac{l}{R} \exp\biggl(-b \frac{R}{l}\biggr)
  \label{eq:slip_correct} \displaybreak[0] \\
  S_{\rm acc} &= 4 \pi R_{\rm 0}^2 N^{\frac{d_{\rm s}}{3}} \left[\left(d_{\rm s} - 2\right) \left(\frac{z}{N}
  \right)^{1 - \gamma} -d_{\rm s} + 3\right] \qquad \text{for } 2 \leq d_{\rm s} \leq 3 \label{eq:S_acc} \displaybreak[0] \\
  S_{\rm acc} &= 4 \pi R_{\rm 0}^2 z^{1 - \gamma} N^{\gamma} \qquad \text{for $d_{\rm f} \leq 2$, $d_{\rm s} = 3$} \label{eq:S_acc_2} \displaybreak[0] \\
  R_{\rm eff} &= \frac{S_{\rm acc}}{4 \pi R_{\rm me,c}} \label{eq:R_eff} \displaybreak[0] \\
  R_{\rm me, c} &= \left(-0.06483 d_{\rm f}^2 + 0.6353 d_{\rm f} - 0.4898\right) R_{\rm geo} \qquad \text{(fit formula)} \label{eq:R_me_c} \displaybreak[0] \\ 
  R_{\rm me, c} &= \frac{\xi \Gamma(d_{\rm f}/2)}{\Gamma((d_{\rm f}-1)/2)} \qquad \text{(analytical solution)} \label{eq:R_me_c_anal} \displaybreak[0] \\ 
  \xi &= R_{\rm geo} \biggl(\frac{2}{d_{\rm f} \Gamma(d_{\rm f}/2)}\biggr)^{\rm 1/d_{\rm f}} \label{eq:xi} \displaybreak[0] \\ 
  R_{\rm me} &= R_{\rm me, c} \frac{C(R_{\rm me})}{C(R_{\rm eff})} \label{eq:R_me} \displaybreak[0] \\ 
%  R_{\rm me}^2 - R_{\rm me,c} C(R_{\rm eff}) R_{\rm me} &= R_{\rm me,c} C(R_{\rm eff}) Q l \exp\biggl(-b \frac{R_{\rm me}}{l}\biggr) 
%  + R_{\rm me,c} C(R_{\rm eff}) A l \label{eq:R_me_expand} \displaybreak[0] \\
  R_{\rm m} &= \left[\frac{3 m(R_{\rm geo})}{4 \pi \rho}\right]^{\frac{1}{3}} \label{eq:R_m} \displaybreak[0] \\
  m(R_{\rm geo}) &= \frac{4 \pi \rho R_{\rm 0}^3}{3 f} R_{\rm geo}^{\ast d_{\rm f}} \label{eq:m_Rgeo} \displaybreak[0] \\
  R_{\rm geo} &= R_{\rm geo}^{\ast} R_{\rm 0} \label{eq:R_geo} \displaybreak[0] \\
  N &= \frac{R_{\rm geo}^{\ast d_{\rm f}}}{f} \label{eq:Num_monomer} \displaybreak[0] \\
  R_{\rm ae}^2 C(R_{\rm ae}) &= \frac{\rho R_{\rm m}^3 C(R_{\rm me})}{\rho_{\rm 0} R_{\rm me}} \displaybreak[0] \\
  R_{\rm ae}^2 C(R_{\rm ae}) &= \frac{\rho R_{\rm m}^3 C(R_{\rm eff})}{\rho_{\rm 0} R_{\rm me,c}} \label{eq:R_ae} \displaybreak[0] 
%  R_{\rm ae}^2 + A l R_{\rm ae} + Q l R_{\rm ae} \exp\biggl(-b \frac{R_{\rm ae}}{l}\biggr) 
%  &= \frac{\rho R_{\rm m}^3 C(R_{\rm eff})}{\rho_{\rm 0} R_{\rm me,c}} \displaybreak[0]
\end{align}

\newpage
\textbf{Note:}
\begin{itemize}
\item Eq. \ref{eq:R_me} builds the basis to convert between $R_{\rm geo}$ and $R_{\rm me}$. 
%Eq. \ref{eq:R_me_expand} is the expanding form of Eq. \ref{eq:R_me}.

\item To convert from $R_{\rm geo}$ (or volume) to $R_{\rm me}$, we use Newton's method to solve the non-linear
equation. The original and first derivative forms of $R_{\rm me}$ are shown in Eq. \ref{eq:f_Rme}
and Eq. \ref{eq:df_Rme}.
\begin{align}
%   f(R_{\rm me}) &= R_{\rm me}^2 - R_{\rm me,c} C(R_{\rm eff}) R_{\rm me} - R_{\rm me,c} C(R_{\rm eff})
%     Q l \exp\biggl(-b \frac{R_{\rm me}}{l}\biggr) - R_{\rm me,c} C(R_{\rm eff}) A l  
%     \label{eq:f_Rme} \displaybreak[0] \\
%   f^ \prime (R_{\rm me}) &= 2 R_{\rm me} - R_{\rm me,c} C(R_{\rm eff}) + R_{\rm me,c} C(R_{\rm eff})
%     Q b \exp\biggl(-b \frac{R_{\rm me}}{l}\biggr) \label{eq:df_Rme} \displaybreak[0]
   f(R_{\rm me}) &= C(R_{\rm eff}) R_{\rm me}^2 - R_{\rm me,c} R_{\rm me} - R_{\rm me,c} 
     Q l \exp\biggl(-b \frac{R_{\rm me}}{l}\biggr) - R_{\rm me,c} A l  
     \label{eq:f_Rme} \displaybreak[0] \\
   f^ \prime (R_{\rm me}) &= 2 C(R_{\rm eff}) R_{\rm me} + R_{\rm me,c}
     Q b \exp\biggl(-b \frac{R_{\rm me}}{l}\biggr) - R_{\rm me,c} \label{eq:df_Rme} \displaybreak[0]
\end{align}

\item To convert from $R_{\rm me}$ to $R_{\rm geo}$ (or volume), we need to first convert $R_{\rm me}$
to $R_{\rm me,c}$, and then use Eq. \ref{eq:R_me_c} or Eq. \ref{eq:R_me_c_anal} to obtain $R_{\rm geo}$.
%The original and first 
%derivative forms of $R_{\rm me,c}$ are shown in Eq. \ref{eq:f_Rmec} and Eq. \ref{eq:df_Rmec}.
%\begin{align}
%   f(R_{\rm me,c}) &= \frac{4 \pi Q l R_{\rm me}}{S_{\rm acc} C(R_{\rm me})} R_{\rm me,c} \exp
%     \biggl(-b \frac{S_{\rm acc}}{4 \pi l R_{\rm me,c}}\biggr) + \biggl(\frac{4 \pi A l R_{\rm me}}{S_{\rm acc} 
%     C(R_{\rm me})} - 1\biggr) R_{\rm me,c} + \frac{R_{\rm me}}{C(R_{\rm me})} 
%     \label{eq:f_Rmec} \displaybreak[0] 
%   f^ \prime (R_{\rm me,c}) &= \frac{4 \pi Q l R_{\rm me}}{S_{\rm acc} C(R_{\rm me})} \exp
%     \biggl(-b \frac{S_{\rm acc}}{4 \pi l R_{\rm me,c}}\biggr) \biggl(1 + b \frac{S_{\rm acc}}{4 \pi l}
%     \frac{1}{R_{\rm me,c}}\biggr) + \frac{4 \pi A l R_{\rm me}}{S_{\rm acc} C(R_{\rm me})} - 1
%     \label{eq:df_Rmec} \displaybreak[0] 
%\end{align}
%where $S_{\rm acc}$ is obtained from $R_{\rm me}$ using Eq. \ref{eq:S_acc_fromRme}.
%where $S_{\rm acc}$ is obtained from Eq. \ref{eq:S_acc}.
%\begin{align}
%  S_{\rm acc} &= 4 \pi R_{\rm 0}^2 z^{1 - \gamma} N^{\gamma} = 4 \pi z^{1 - \gamma} R_{\rm me}^2
%    \label{eq:S_acc_fromRme} \displaybreak[0]
%\end{align}

From Eq. \ref{eq:S_acc_2}, we know for $d_{\rm f} \leq 2$, $d_{\rm s} = 3$ (currently in our simulation
we use $d_{\rm f} = 2$), we could get a simplified form for $S_{\rm acc}$. Thus, 
together with Eq. \ref{eq:Num_monomer}, Eq. \ref{eq:R_eff} could be re-written as 

\begin{equation}
   R_{\rm eff} = \frac{R_{\rm 0}^2 N^{\rm \gamma}}{R_{\rm me,c}} = \frac{R_{\rm 0}^{\rm 2-d_{\rm f} \gamma}}
        {f^{\rm \gamma} h_{\rm KR}^{\rm d_{\rm f} \gamma}} R_{\rm me,c}^{\rm d_{\rm f} \gamma - 1}
     \label{eq:R_eff_simplified}
\end{equation}

where 

\begin{equation} \label{eq:h_KR}
  h_{\rm KR} = -0.06483 d_{\rm f}^2 + 0.6353 d_{\rm f} - 0.4898
\end{equation}

is the Kirkwood-Riseman ratio. If we define a constant 

\begin{equation} \label{eq:phi}
  \phi = \frac{R_{\rm 0}^{\rm 2-d_{\rm f} \gamma}}
        {f^{\rm \gamma} h_{\rm KR}^{\rm d_{\rm f} \gamma}}
\end{equation}

so $R_{\rm eff}$ could be written as 

\begin{equation} \label{eq:R_eff_simplified_2}
  R_{\rm eff} = \phi R_{\rm me,c}^{\rm d_{\rm f} \gamma - 1}
\end{equation}

Therefore, using Eq. \ref{eq:R_me} we are able to write the original and first derivative forms of
$R_{\rm me,c}$ in Eq. \ref{eq:f_Rme_c} and Eq. \ref{eq:df_Rme_c}.

\begin{align}  
   f(R_{\rm me,c}) &= C(R_{\rm me}) R_{\rm me,c} - \frac{A l R_{\rm me}}{\phi} R_{\rm me,c}^{\rm 1-d_{\rm f} \gamma}
     - \frac{Q l R_{\rm me}}{\phi} R_{\rm me,c}^{\rm 1-d_{\rm f} \gamma} \exp\biggl(- \frac{b \phi}{l}
     R_{\rm me,c}^{\rm d_{\rm f} \gamma - 1}\biggr) - R_{\rm me}
     \label{eq:f_Rme_c} \displaybreak[0] \\
   f^ \prime (R_{\rm me,c}) &= C(R_{\rm me}) - \frac{l R_{\rm me}}{\phi} (1 - d_{\rm f} \gamma)
     R_{\rm me,c}^{\rm -d_{\rm f} \gamma} \left[A + Q \exp\biggl(- \frac{b \phi}{l} R_{\rm me,c}^{\rm d_{\rm f} \gamma - 1}
     \biggr)\right] - Q b R_{\rm me} (1 - d_{\rm f} \gamma) R_{\rm me,c}^{\rm -1} 
     \exp\biggl(- \frac{b \phi}{l} R_{\rm me,c}^{\rm d_{\rm f} \gamma -1}\biggr)
     \label{eq:df_Rme_c} \displaybreak[0]
\end{align}

Note that Eq. \ref{eq:f_Rme_c} and Eq. \ref{eq:df_Rme_c} are only available when $d_{\rm f} \leq 2$, $d_{\rm s} = 3$.
When $d_{\rm f}$ is greater than 2, $d_{\rm s}$ will be $6/d_{\rm f}$, we will not be able to
obtain a simplified equation for $R_{\rm eff}$ as shown in Eq. \ref{eq:R_eff_simplified_2}.

\end{itemize}
\end{document}
