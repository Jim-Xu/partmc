\documentclass{article}
\usepackage[margin=3cm]{geometry}
\usepackage{amsmath}

\begin{document}

\section{Variables for aqueous chemistry reactions}

For a single reaction the variables are:

\begin{tabular}{ll}
  $R_f$ & forward rate \\
  $R_b$ & backward rate \\
  $C$ & conversion factor (only for \texttt{HENRY} reactions) \\
  $Y_i$ & yield factor (1, 2, \ldots) for product $i$ \\
  $w_i$ & mixing ratio for species $i$
\end{tabular}

\section{Aqueous chemistry reaction rates}

For a single reaction, the production rate of reactant $i$ is:
\begin{equation}
  \dot{w}_i^{\text{reactant}} = C(R_b - R_r)
\end{equation}
and the production rate of product $i$ is
\begin{equation}
  \dot{w}_i^{\text{product}} = Y_i R_f - R_b.
\end{equation}
These rates are summed over all reactions for reactants and products
to give the total rate for each species.

\section{Reaction-string format}

Reactions are output as strings like:
\begin{verbatim}
 88: AQUA: HSO5m(76) + HSO3m(102) + Hp(108) ==> 2*SO4mm(97) + 3*Hp(108) ; rate TEMP3:: 7140000, 0
\end{verbatim}
This means that reaction number \texttt{88} is of type \texttt{AQUA}
and has reactants \texttt{HSO5m}, \texttt{HSO3m}, and \texttt{Hp}. It
has products \texttt{BRm} (with a yield of $Y_1 = 2$) and \texttt{Hp}
(with a yield of $Y_2 = 3$). The information after \texttt{rate}
indicates the type of rate function and the parameters for it.

\section{Notes}

We don't have ``yield'' factors for reactants, but they can be
repeated like in:
\begin{verbatim}
 123: AQUA: O2CH2COOm(55) + O2CH2COOm(55) ==> 2*CHOH2COOm(32) + aH2O2(98) ; rate TEMP3:: 20000000, 0
\end{verbatim}

A single species can appear as both a product and reactant in the same
reaction, so its total production rate for that reaction will be the
sum of its product rate and reactant rate.

\end{document}
